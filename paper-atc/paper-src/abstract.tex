Blockchain systems are used to record transactions among parties without a central authority. Since the very first instance of a blockchain systems, Bitcoin, tremendous amount of blockchain systems with various consensus protocols are designed and implemented to achieve fast transaction rate. With the transaction rate increasing, experiments and studies show that network layer become the bottleneck on the way to further improve blockchain system efficiency. However, current blockchain systems are based on unstructured, structured, or hybrid Peer-to-Peer networks. Gossiping messages on such networks creates repeated messages and leads to traffic congestion when transaction rates grows. In addition, lack of attention paid to geographical locality and security in the network layer also limits the improvement of the P2P network underlying blockchain systems.

In this paper, we present \textbf{H}idden \textbf{G}eographical \textbf{F}ractal \textbf{R}andom \textbf{R}ing (HGFRR). It constructs and maintains a recursive DHT-ring like structure according to geographical proximity of nodes. The broadcast operation on such a network which contains $N$ nodes achieves $O(N)$ in terms of message complexity, and $O(logN)$ in terms of time complexity. Security issues are also addressed to protect the anonymity of nodes. Evaluation shows that HGFRR outperforms typical P2P network in blockchain systems in both time complexity and messages complexity. Consequently, the throughput of the blockchain system can be further improved around 1.4X to 2.1X. Source code of the implementation of HGFRR (in C++) will be available on GitHub once appropriate.